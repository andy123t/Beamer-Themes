%%%%%%%%%%%%%%%%%%%%%%%%%%%%%%%%%%%%%%%%%
% Beamer Presentation
% LaTeX Template
%%%%%%%%%%%%%%%%%%%%%%%%%%%%%%%%%%%%%%%%%

%----------------------------------------------------------------------------------------
%	PACKAGES AND THEMES
%----------------------------------------------------------------------------------------

%\documentclass[notheorems,compress]{beamer}
\documentclass[notheorems]{beamer}

\mode<presentation> {

% The Beamer class comes with a number of default slide themes
% which change the colors and layouts of slides. Below this is a list
% of all the themes, uncomment each in turn to see what they look like.

\setbeamertemplate{background canvas}[vertical shading][bottom=red!10,top=blue!10]
\setbeamertemplate{blocks}[rounded][shadow=true]

%\usetheme{default}
%\usetheme{CambridgeUS}
%\usetheme{Madrid}
%\usetheme{Singapore}
\usetheme{Warsaw}


\setbeamertemplate{theorems}[numbered]
%\setbeamercovered{transparent}
\usefonttheme[onlymath]{serif}
%\usetheme{boxes}
%\setbeamertemplate{footline}{} % To remove the footer line in all slides uncomment this line
%\setbeamertemplate{footline}[page number]
\setbeamertemplate{navigation symbols}{}
}


%\usepackage{amsmath}
%\usepackage{graphicx} % Allows including images
\usepackage{booktabs} % Allows the use of \toprule, \midrule and \bottomrule in tables

\usepackage[UTF8,noindent]{ctex}
\usepackage[english]{babel} %show english numerate
\usepackage{amsmath,amssymb,version}
\usepackage{graphicx,fancybox,mathrsfs,multirow}
\usepackage{booktabs}
\usepackage{epsfig,epstopdf}
\usepackage{url,hyperref}
\usepackage{tabularx,array,makecell}
\usepackage{color,xcolor}
\usepackage{cases}
\usepackage{mathtools}
\usepackage{zhlipsum}


\setbeamertemplate{footline}%{shadow theme}
{%
  \leavevmode%
  \hbox{\begin{beamercolorbox}[wd=.5\paperwidth,ht=2.5ex,dp=1.125ex,leftskip=.3cm plus1fil,rightskip=.3cm]{author in head/foot}%
    \usebeamerfont{author in head/foot}\insertshortauthor
  \end{beamercolorbox}%
  \begin{beamercolorbox}[wd=.5\paperwidth,ht=2.5ex,dp=1.125ex,leftskip=.3cm,rightskip=.3cm plus1fil]{title in head/foot}%
    \usebeamerfont{title in head/foot} \hfill \insertshorttitle \hfill \insertframenumber\,/\,\inserttotalframenumber%
  \end{beamercolorbox}}%
  \vskip0pt%
}

%---------- 定理设置  --------------
\setbeamertemplate{theorems}[numbered]
%\newtheorem{theorem}{Theorem}
\newtheorem{theorem}{定理}
\numberwithin{theorem}{section}
%\newtheorem{definition}{Definition}
\newtheorem{definition}{定义}
\numberwithin{definition}{section}
%\newtheorem{lemma}{Lemma}
\newtheorem{lemma}{引理}
\numberwithin{lemma}{section}
%\newtheorem{proposition}{Proposition}
\newtheorem{proposition}{命题}
\numberwithin{proposition}{section}
%\newtheorem{corollary}{Corollary}
\newtheorem{corollary}{引理}
\numberwithin{corollary}{section}
\theoremstyle{example}
%\newtheorem{example}{Example}
\newtheorem{example}{例}
%\numberwithin{example}{section}
\renewcommand{\proofname}{\heiti 证明}

\setbeamertemplate{caption}[numbered]
\numberwithin{figure}{section}
\numberwithin{table}{section}
\numberwithin{equation}{section}

%\setbeamerfont{normal text}{family=\songti}
%\setbeamerfont{frametitle}{family=\large\bfseries}
%\setbeamerfont{title}{family=\bfseries}
%\setbeamerfont{subtitle}{family=\kaishu}



%\AtBeginSection[]
%{ \begin{frame}
%    \frametitle{目录}
%    \tableofcontents[currentsection,currentsubsection]
%  \end{frame}
%  \addtocounter{framenumber}{-1}  %目录页不计算页码
%}

\AtBeginSection[]
{ \begin{frame}
    \frametitle{目录}
    \tableofcontents[currentsection,currentsubsection] %hideallsubsections
  \end{frame}
  \addtocounter{framenumber}{-1}  %目录页不计算页码
}

%\AtBeginSubsection[]
%{
%	\begin{frame}%[shrink]
%    \frametitle{目录}
%	%\thispagestyle{empty}
%	\addtocounter{framenumber}{-1}
%	\tableofcontents[
%	sectionstyle=show/shaded,
%	subsectionstyle=show/shaded/hide]
%\end{frame}
%}
%----------------------------------------------------------------------------------------
%	TITLE PAGE
%----------------------------------------------------------------------------------------

\title[Short 题目]{Full Title of the Talk 题目} % The short title appears at the bottom of every slide, the full title is only on the title page

\author{姓名}  % 名字
\institute[NU] % 机构缩写
{
Name of University \\ % 机构全称
\medskip
\textit{name@email.com} % 邮件
}
\date[2020.6.23]{2020~年~6~月~23~日} % 日期

\graphicspath{{./Figures/}}

\begin{document}
\songti

%\thispagestyle{empty}
\begin{frame}
\titlepage % Print the title page as the first slide
\end{frame}

\begin{frame}
\frametitle{目录} % Table of contents slide, comment this block out to remove it
\tableofcontents %[hideallsubsections] % Throughout your presentation, if you choose to use \section{} and \subsection{} commands, these will automatically be printed on this slide as an overview of your presentation
\end{frame}

%----------------------------------------------------------------------------------------
%	PRESENTATION SLIDES
%----------------------------------------------------------------------------------------

%------------------------------------------------
\section{文本与 Block}
%------------------------------------------------

\begin{frame}{文本测试}
这是一段测试文字。这是一段测试文字。这是一段测试文字。这是一段测试文字。这是一段测试文字。这是一段测试文字。
这是一段测试文字。这是一段测试文字。这是一段测试文字。这是一段测试文字。这是一段测试文字。这是一段测试文字。

\vspace{1ex}
这是一段测试文字。这是一段测试文字。这是一段测试文字。这是一段测试文字。这是一段测试文字。这是一段测试文字。
这是一段测试文字。这是一段测试文字。这是一段测试文字。这是一段测试文字。这是一段测试文字。这是一段测试文字。

\end{frame}

%-----------------------------------------------

\begin{frame}
\frametitle{Blocks of Highlighted Text}
\begin{block}{Block Title}
This is the block environment. The quick brown fox jumps over the lazy dog. The quick brown fox jumps over the lazy dog. The quick brown fox jumps over the lazy dog.
\end{block}

\begin{exampleblock}{Block Title}
This is the exampleblock environment. The quick brown fox jumps over the lazy dog. The quick brown fox jumps over the lazy dog.
\end{exampleblock}

\begin{alertblock}{Block Title}
This is the alertblock environment. The quick brown fox jumps over the lazy dog. The quick brown fox jumps over the lazy dog.
\end{alertblock}
\end{frame}

%------------------------------------------------
\section{列表环境与分栏}

\begin{frame}
\frametitle{列表环境}

计数列表环境
\begin{enumerate}
\item 这是一个计数列表环境.
\item 这是一个计数列表环境.
\item 这是一个计数列表环境.
\end{enumerate}

\vspace{2ex}
不计数列表环境
\begin{itemize}[<+-| alert@+>]
\item 这是一个不计数列表环境.
\item 这是一个不计数列表环境.
\item 这是一个不计数列表环境.
\end{itemize}

\end{frame}

%------------------------------------------------

\begin{frame}
\frametitle{左右分栏}  % Multiple Columns
\begin{columns}[t] % The "c" option specifies centered vertical alignment while the "t" option is used for top vertical alignment

\column{.45\textwidth} % Left column and width
\textbf{Heading}
\begin{enumerate}
\item Statement
\item Explanation
\item Example
\end{enumerate}

\column{.5\textwidth} % Right column and width
The quick brown fox jumps over the lazy dog. The quick brown fox jumps over the lazy dog. The quick brown fox jumps over the lazy dog. The quick brown fox jumps over the lazy dog.
\end{columns}
\end{frame}

%------------------------------------------------
\section{定理与表格}
%------------------------------------------------

\begin{frame}
\frametitle{定理环境}
\begin{definition} \upshape
This is a definition environment. 这是一个定义环境.
\end{definition}


\begin{lemma} \upshape
This is a lemma environment. 这是一个引理环境.
\end{lemma}

\begin{proposition} \upshape
This is a proposition environment. 这是一个命题环境.
\end{proposition}


\begin{theorem}[Mass--energy] \upshape
This is a theorem environment. 这是一个定理环境.
\end{theorem}

\end{frame}

%------------------------------------------------


\begin{frame}
\frametitle{定理示例}

\begin{lemma}[Lax-Milgram Lemma] \upshape
Let $X$ be a Hilbert space, let $a(\cdot, \cdot)$ : $X \times X \rightarrow \mathbb{R}$ be a continuous and coercive bilinear form, and let $F : X \rightarrow \mathbb{R}$ be a linear functional in $X^{\prime}$. Then the variational problem:
\begin{equation}
  \alert{
  \left\{\begin{aligned}
  &\text {Find } u \in X \text { such that } \\
  &a(u, v)=F(v), \forall v \in X
  \end{aligned} \right. }
\end{equation}
has a unique solution. Moreover, we have

\begin{equation}
  \alert{ \|u\| \leq \frac{1}{\alpha}\|F\|_{X^{\prime}}  }
\end{equation}
\end{lemma}

\end{frame}

%------------------------------------------------

\begin{frame}[fragile] % Need to use the fragile option when verbatim is used in the slide
\frametitle{Verbatim}
\begin{example}[Theorem Slide Code]
\begin{verbatim}
\begin{frame}
\frametitle{Theorem}
\begin{theorem}[Mass--energy equivalence]
$E = mc^2$
\end{theorem}
\end{frame}\end{verbatim}
\end{example}

\begin{table}
\caption{这是一个三线表.}
\begin{tabular}{l l l}
\toprule
\textbf{Treatments} & \textbf{Response 1} & \textbf{Response 2}\\
\midrule
Treatment 1 & 0.0003262 & 0.562 \\
Treatment 2 & 0.0015681 & 0.910 \\
\bottomrule
\end{tabular}
\end{table}

\end{frame}


%------------------------------------------------

\section{插图环境}

\begin{frame}
\frametitle{插图环境}

Uncomment the code on this slide to include your own image from the same directory as the template .TeX file.
\begin{figure}[htp!]
\centering
\includegraphics[width=0.5\linewidth]{image}
\caption{Caption of Figure 1.} \label{fig:A}
\end{figure}
\end{frame}


%------------------------------------------------
\section{参考文献}
%------------------------------------------------

\begin{frame}[fragile] % Need to use the fragile option when verbatim is used in the slide
\frametitle{Citation}
An example of the \verb|\cite| command to cite within the presentation:\\~

This statement requires citation \cite{Smith2012}. \\~

文献引用示例 \cite{LiLiu1997}, 可以修改引用文献样式.
\end{frame}

%------------------------------------------------

\begin{frame}
\frametitle{References}
\footnotesize{
\begin{thebibliography}{99} % Beamer does not support BibTeX so references must be inserted manually as below
\bibitem[Smith, 2012]{Smith2012} John Smith, Title of the publication, \emph{Journal Name}, 12(3):45--678, 2012.
\bibitem[李荣华, 1997]{LiLiu1997} 李荣华, 刘播. 微分方程数值解法. 东南大学出版社, 1997.
\end{thebibliography}
}
\end{frame}

%------------------------------------------------
\thispagestyle{empty}
%\setbeamertemplate{background canvas}[vertical shading][bottom=white,top=structure.fg!25]
%\setbeamertemplate{footline}{}
%\setbeamertemplate{headline}{}
\begin{frame}{}
\Huge{\centerline{The End}}
\end{frame}

%----------------------------------------------------------------------------------------

\end{document}

