%%%%%%%%%%%%%%%%%%%%%%%%%%%%%%%%%%%%%%%%%
% Beamer Presentation
% LaTeX Template
%%%%%%%%%%%%%%%%%%%%%%%%%%%%%%%%%%%%%%%%%

%----------------------------------------------------------------------------------------
%	PACKAGES AND THEMES
%----------------------------------------------------------------------------------------

%\documentclass[compress]{beamer}
\documentclass{beamer}

\mode<presentation> {

% The Beamer class comes with a number of default slide themes
% which change the colors and layouts of slides. Below this is a list
% of all the themes, uncomment each in turn to see what they look like.

\setbeamertemplate{background canvas}[vertical shading][bottom=red!10,top=blue!10]
\setbeamertemplate{blocks}[rounded][shadow=true]

%\usetheme{default}
%\usetheme{CambridgeUS}
%\usetheme{Madrid}
%\usetheme{Singapore}
\usetheme{Warsaw}


\setbeamertemplate{theorems}[numbered]
%\setbeamercovered{transparent}
\usefonttheme[onlymath]{serif}
%\usetheme{boxes}
%\setbeamertemplate{footline}{} % To remove the footer line in all slides uncomment this line
%\setbeamertemplate{footline}[page number]
\setbeamertemplate{navigation symbols}{}
}


%\usepackage{amsmath}
%\usepackage{graphicx} % Allows including images
\usepackage{booktabs} % Allows the use of \toprule, \midrule and \bottomrule in tables

%\usepackage[UTF8,fntef]{ctexcap}
\usepackage[UTF8,noindent]{ctex}
\usepackage[english]{babel} %show english numerate
%\usepackage[adobefonts,punct,UTF8,indent,fancyhdr]{ctexcap}
%\usepackage{CJKutf8}
\usepackage{epsfig,amssymb,amsmath,version}
\usepackage{amssymb,version,graphicx,fancybox,mathrsfs,multirow}
\usepackage{epstopdf}
\usepackage{url,hyperref}

\usepackage{color,xcolor}
\usepackage{cases}
\usepackage{mathtools}
\usepackage{zhlipsum}
%\DeclarePairedDelimiter{\ceil}{\lceil}{\rceil}
%\usepackage[UTF8,noindent]{ctex}

\setbeamertemplate{footline}%{shadow theme}
{%
  \leavevmode%
  \hbox{\begin{beamercolorbox}[wd=.5\paperwidth,ht=2.5ex,dp=1.125ex,leftskip=.3cm plus1fil,rightskip=.3cm]{author in head/foot}%
    \usebeamerfont{author in head/foot}\insertshortauthor
  \end{beamercolorbox}%
  \begin{beamercolorbox}[wd=.5\paperwidth,ht=2.5ex,dp=1.125ex,leftskip=.3cm,rightskip=.3cm plus1fil]{title in head/foot}%
    \usebeamerfont{title in head/foot} \hfill \insertshorttitle \hfill \insertframenumber\,/\,\inserttotalframenumber%
  \end{beamercolorbox}}%
  \vskip0pt%
}

\setbeamertemplate{theorems}[numbered]
\newtheorem{thm}{Theorem}
\numberwithin{thm}{section}
\newtheorem{defn}{Definition}
\numberwithin{defn}{section}
\newtheorem{lmm}{Lemma}
\numberwithin{lmm}{section}
\newtheorem{pro}{Proof}
\theoremstyle{example}
\newtheorem{exam}{Example}
%\numberwithin{exam}{section}

\setbeamertemplate{caption}[numbered]
\numberwithin{figure}{section}
\numberwithin{table}{section}
\numberwithin{equation}{section}

%\setbeamerfont{normal text}{family=\songti}
%\setbeamerfont{frametitle}{family=\large\bfseries}
%\setbeamerfont{title}{family=\bfseries}
%\setbeamerfont{subtitle}{family=\kaishu}



%\AtBeginSection[]
%{ \begin{frame}
%    \frametitle{目录}
%    \tableofcontents[currentsection,currentsubsection]
%  \end{frame}
%  \addtocounter{framenumber}{-1}  %目录页不计算页码
%}

\AtBeginSection[]
{ \begin{frame}
    \frametitle{目录}
    \tableofcontents[currentsection,currentsubsection] %hideallsubsections
  \end{frame}
  \addtocounter{framenumber}{-1}  %目录页不计算页码
}

%\AtBeginSubsection[]
%{
%	\begin{frame}%[shrink]
%    \frametitle{目录}
%	%\thispagestyle{empty}
%	\addtocounter{framenumber}{-1}
%	\tableofcontents[
%	sectionstyle=show/shaded,
%	subsectionstyle=show/shaded/hide]
%\end{frame}
%}
%----------------------------------------------------------------------------------------
%	TITLE PAGE
%----------------------------------------------------------------------------------------

\title[Short 题目]{Full Title of the Talk 题目} % The short title appears at the bottom of every slide, the full title is only on the title page

\author{姓名} % Your name
\institute[大学] % Your institution as it will appear on the bottom of every slide, may be shorthand to save space
{
大学全称 \\ % Your institution for the title page
\medskip
\textit{john@smith.com} % Your email address
}
\date{\today} % Date, can be changed to a custom date

\graphicspath{{./Figures/}}

\begin{document}
\songti

%\thispagestyle{empty}
\begin{frame}
\titlepage % Print the title page as the first slide
\end{frame}

\begin{frame}
\frametitle{目录} % Table of contents slide, comment this block out to remove it
\tableofcontents %[hideallsubsections] % Throughout your presentation, if you choose to use \section{} and \subsection{} commands, these will automatically be printed on this slide as an overview of your presentation
\end{frame}

%----------------------------------------------------------------------------------------
%	PRESENTATION SLIDES
%----------------------------------------------------------------------------------------

%------------------------------------------------
\section{第一节} % Sections can be created in order to organize your presentation into discrete blocks, all sections and subsections are automatically printed in the table of contents as an overview of the talk
%------------------------------------------------

\subsection{第一节第一小节} % A subsection can be created just before a set of slides with a common theme to further break down your presentation into chunks

\begin{frame}{文本测试}
\zhlipsum[1]
\end{frame}

%------------------------------------------------

\begin{frame}
\frametitle{Bullet Points}
\begin{itemize}
\item Lorem ipsum dolor sit amet, consectetur adipiscing elit
\item Aliquam blandit faucibus nisi, sit amet dapibus enim tempus eu
\item Nulla commodo, erat quis gravida posuere, elit lacus lobortis est, quis porttitor odio mauris at libero
\item Nam cursus est eget velit posuere pellentesque
\item Vestibulum faucibus velit a augue condimentum quis convallis nulla gravida
\end{itemize}
\end{frame}

%------------------------------------------------
\subsection{第一节第二小节}
\begin{frame}
\frametitle{Blocks of Highlighted Text}
\begin{block}{Block 1}
Lorem ipsum dolor sit amet, consectetur adipiscing elit. Integer lectus nisl, ultricies in feugiat rutrum, porttitor sit amet augue.
\end{block}

\begin{exampleblock}{Block 2}
Pellentesque sed tellus purus. Class aptent taciti sociosqu ad litora torquent per conubia nostra, per inceptos himenaeos.
\end{exampleblock}

\begin{alertblock}{Block 3}
Suspendisse tincidunt sagittis gravida. Curabitur condimentum, enim sed venenatis rutrum, ipsum neque consectetur orci, sed blandit justo nisi ac lacus.
\end{alertblock}
\end{frame}

%------------------------------------------------

\begin{frame}
\frametitle{Multiple Columns}
\begin{columns}[c] % The "c" option specifies centered vertical alignment while the "t" option is used for top vertical alignment

\column{.45\textwidth} % Left column and width
\textbf{Heading}
\begin{enumerate}
\item Statement
\item Explanation
\item Example
\end{enumerate}

\column{.5\textwidth} % Right column and width
Lorem ipsum dolor sit amet, consectetur adipiscing elit. Integer lectus nisl, ultricies in feugiat rutrum, porttitor sit amet augue. Aliquam ut tortor mauris. Sed volutpat ante purus, quis accumsan dolor.

\end{columns}
\end{frame}

%------------------------------------------------
\section{第二节}
%------------------------------------------------
\subsection{第二节第一小节}
\begin{frame}
\frametitle{表格和引理}
\begin{table}
\caption{Table caption}
\begin{tabular}{l l l}
\toprule
\textbf{Treatments} & \textbf{Response 1} & \textbf{Response 2}\\
\midrule
Treatment 1 & 0.0003262 & 0.562 \\
Treatment 2 & 0.0015681 & 0.910 \\
Treatment 3 & 0.0009271 & 0.296 \\
\bottomrule
\end{tabular}
\end{table}
\begin{lmm}
  For any $v \in H_{A}^{r}(\Lambda)$ and $r \geq 0$,
  \begin{equation}
    \|P_{N} v-v\| \leq c N^{-r}\|v\|_{r, A}.
  \end{equation}
\end{lmm}
\end{frame}

%------------------------------------------------

\begin{frame}
\frametitle{定理}

\begin{thm}[Lax-Milgram Lemma]
Let $X$ be a Hilbert space, let $a(\cdot, \cdot)$ : $X \times X \rightarrow \mathbb{R}$ be a continuous and coercive bilinear form, and let $F : X \rightarrow \mathbb{R}$ be a linear functional in $X^{\prime}$. Then the variational problem:
\begin{equation}
  \alert{
  \left\{\begin{aligned}
  &\text {Find } u \in X \text { such that } \\
  &a(u, v)=F(v), \forall v \in X
  \end{aligned} \right. }
\end{equation}
has a unique solution. Moreover, we have

\begin{equation}
  \alert{ \|u\| \leq \frac{1}{\alpha}\|F\|_{X^{\prime}}  }
\end{equation}
\end{thm}

\end{frame}


\subsection{第二节第二小节}
%------------------------------------------------

\begin{frame}[fragile] % Need to use the fragile option when verbatim is used in the slide
\frametitle{Verbatim}
\begin{exam}[Theorem Slide Code]
\begin{verbatim}
\begin{frame}
\frametitle{Theorem}
\begin{theorem}[Mass--energy equivalence]
$E = mc^2$
\end{theorem}
\end{frame}\end{verbatim}
\end{exam}

\begin{thm}[Mass--energy 等价]
$E = mc^2$
\end{thm}

\end{frame}

%------------------------------------------------




\begin{frame}
\frametitle{Figure}

Uncomment the code on this slide to include your own image from the same directory as the template .TeX file.
\begin{figure}
\includegraphics[width=0.8\linewidth]{Image.eps}
\end{figure}
\end{frame}

%------------------------------------------------

\begin{frame}[fragile] % Need to use the fragile option when verbatim is used in the slide
\frametitle{Citation}
An example of the \verb|\cite| command to cite within the presentation:\\~

This statement requires citation \cite{p1}.
\end{frame}

%------------------------------------------------

\begin{frame}
\frametitle{References}
\footnotesize{
\begin{thebibliography}{99} % Beamer does not support BibTeX so references must be inserted manually as below
\bibitem[Smith, 2012]{p1} John Smith (2012)
\newblock Title of the publication
\newblock \emph{Journal Name} 12(3), 45 -- 678.
\end{thebibliography}
}
\end{frame}

%------------------------------------------------
\thispagestyle{empty}
%\setbeamertemplate{background canvas}[vertical shading][bottom=white,top=structure.fg!25]
%\setbeamertemplate{footline}{}
%\setbeamertemplate{headline}{}
\begin{frame}{}
\Huge{\centerline{The End}}
\end{frame}

%----------------------------------------------------------------------------------------

\end{document}

